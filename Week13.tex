\documentclass{article}
\usepackage[english]{babel}
\usepackage[utf8]{inputenc}
\usepackage{fancyhdr}
\usepackage{geometry}
\usepackage{enumitem}
\usepackage{amsmath}
\usepackage{graphicx}
\usepackage{tcolorbox}
\usepackage{amssymb}
\usepackage[thinc]{esdiff}

\geometry{letterpaper, portrait, margin=1in}
\pagestyle{fancy}
\fancyhf{}
\lhead{Keerthik Muruganandam}
\rhead{Written Work 13}

\begin{document}

\begin{enumerate}[label=\textbf{(13.\arabic*)}]

\item If $a$ and $b$ are positive numbers, find the maximum value of $f(x)=x^a(1-x)^b$, $0\le x\le 1$.

To find critical points, we will take the derivative of the function $f(x)=x^a(1-x)^b$, and solve for 0.
\begin{align*}
f^\prime(x)&=\diff{ }{x}\left(x^a(1-x)^b\right) \\
&=\diff{ }{x}\left(x^a\right)(1-x)^b+\diff{ }{x}\left((1-x)^b\right)x^a \\
&=(ax^{a-1})(1-x)^b+b(1-x)^{b-1}\diff{ }{x}(1-x)x^a \\
&=-x^ab(1-x)^{b-1}+a(x-1)^bx^{a-1} \\
&=x^{a-1}(1-x)^{b-1}[(1-x)a-bx] \\
&=x^{a-1}(1-x)^{b-1}[a-x(a+b)] \\
\end{align*}
Now to find the critical points, we must solve for $f^\prime=0$.
\[x^{a-1}(1-x)^{b-1}[a-x(a+b)]=0\]
From here we can see that x must either equal 0, 1 , or $\dfrac{a}{a+b}$. Because we are trying to find the maximum on the interval [0,1], the critical points must be on the open interval (0,1). Thus, we can eliminate $x=0$ and $x=1$ from the pool of possible critical points, leaving $x=\dfrac{a}{a+b}$ as the only possible critical point on the interval.\\
Now that we have our critical point, we can just solve $f(\dfrac{a}{a+b})$ to find the possible maximum of the function in the interval $0\le x\le 1$.
\begin{align*}
f\left(\frac{a}{a+b}\right)&={\left(\frac{a}{a+b}\right)}^a{\left(1-\frac{a}{a+b}\right)}^b \\
&={\left(\frac{a}{a+b}\right)}^a{\left(\frac{b}{a+b}\right)}^b \\
&= \frac{a^a}{{\left(a+b\right)}^a}\cdot\frac{b^b}{{\left(a+b\right)}^b} \\
&=\frac{a^ab^b}{{\left(a+b\right)}^{a+b}}
\end{align*}
To determine whether $\dfrac{a}{a+b}$ is a maximum, we must check if $f(0) \text{ and } f(1)$ are greater than $f\left(\dfrac{a}{a+b}\right)$.
\begin{align*}
f(0)&=0^a(1-0)^b \\
&=0\cdot1^b \\
&=0 \\
f(1)&=1^a(1-1)^b \\
&=1^a\cdot0
&=0
\end{align*}
Now we know that $\dfrac{a}{a+b}$ is the maximum point because $a \text{ and } b$ are positive numbers and thus are greater than 0.
\begin{tcolorbox}[colback=white]
The maximum value of  $f(x)=x^a(1-x)^b$, $0\le x\le 1$ is $\dfrac{a^ab^b}{{\left(a+b\right)}^{a+b}}$
\centering
\end{tcolorbox}

\par
Next Page.$ \rightarrow$


\end{enumerate}


\end{document}