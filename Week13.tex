\documentclass{article}
\usepackage[english]{babel}
\usepackage[utf8]{inputenc}
\usepackage{fancyhdr}
\usepackage{geometry}
\usepackage{enumitem}
\usepackage{amsmath}
\usepackage{graphicx}
\usepackage{tcolorbox}
\usepackage{amssymb}
\usepackage[thinc]{esdiff}

\geometry{letterpaper, portrait, margin=1in}
\pagestyle{fancy}
\fancyhf{}
\lhead{Keerthik Muruganandam}
\rhead{Written Work 13}

\begin{document}

\begin{enumerate}[label=\textbf{(13.\arabic*)}]
    \item If $a$ and $b$ are positive numbers, find the maximum value of $f(x)=x^a(1-x)^b$, $0\le x\le 1$.
    
    We will complete the first derivative test to find the Critical Points
\begin{align*}
f^\prime(x)&=\diff{ }{x}\left(x^a(1-x)^b\right) \\
&=\diff{ }{x}\left(x^a\right)(1-x)^b+\diff{ }{x}\left((1-x)^b\right)x^a \\
&=(ax^{a-1})(1-x)^b+b(1-x)^{b-1}\diff{ }{x}(1-x)x^a \\
&=-x^ab(1-x)^{b-1}+a(x-1)^bx^{a-1} \\
&=x^{a-1}(1-x)^{b-1}[(1-x)a-bx] \\
&=x^{a-1}(1-x)^{b-1}[a-x(a+b)] \\
\end{align*}
Now to find the critical points, we must solve for $f^\prime=0$.
\[x^{a-1}(1-x)^{b-1}[a-x(a+b)=0\]
For $f^\prime(x)=0$ to be true, either $x^{a-1}$, $(x-1)^{b-1}$, or $a-x(a+b)$ must equal 0. 
\[\therefore x=0\text{ or }x=1\text{ or }x=\frac{a}{a+b}\text{, }a+b\neq0\]

\end{enumerate}


\end{document}