\documentclass{article}
\usepackage[english]{babel}
\usepackage[utf8]{inputenc}
\usepackage{fancyhdr}
\usepackage{geometry}
\usepackage{enumitem}
\usepackage{amsmath, amssymb}
\usepackage{graphicx}
\usepackage{float}
\usepackage{esdiff}

\geometry{letterpaper, portrait, margin=1in}
\graphicspath{ {images/} }
\pagestyle{fancy}
\fancyhf{}
\lhead{Keerthik Muruganandam}
\rhead{8/15/21 Summer Assignment}

\begin{document}

\begin{enumerate}[label=\textbf{\arabic*.}]
\item Show that $y=\dfrac{2}{3}e^x+e^{-2x}$ is a solution of the differential equation $y^\prime+2y=2e^x$.\\


\vspace{5pt}
First, let us define $y=\dfrac{2}{3}e^x+e^{-2x}$ as Equation 1, or $(1)$, $y^\prime+2y=2e^x$ as Equation 2, or $(2)$. Next, the value of $y^\prime$ needs to be found. Because both terms in the function are very similar to the natural exponential function $e^x$, this function is relatively simple to differentiate.
\begin{align*}
y &= \frac{2}{3}e^x+e^{-2x}\\
y^\prime &= \frac{2}{3}\diff{}{x}(x)e^x+\diff{ }{x}(-2x)e^{-2x}\\
&= \frac{2}{3}e^x-2e^{-2x}
\end{align*}

Substituting the values for $y^\prime$ and $y$ that we found in the first step into Equation 2 provides the new equation:
\[\frac{2}{3}e^x-2e^{-2x} + 2\left(\frac{2}{3}e^x+e^{-2x}\right) =  2e^x\] 
To verify this equation, all that is needed is the simplification of the left hand side, which is done below.
\begin{align*}
\frac{2}{3}e^x-2e^{-2x} + 2\left(\frac{2}{3}e^x+e^{-2x}\right) &= \frac{2}{3}e^x-2e^{-2x} +  \frac{4}{3}e^x+2e^{-2x}\\
&= \frac{6}{3}e^x\\
&= 2e^x
\end{align*}
Thus, it has been shown that $y=\dfrac{2}{3}e^x+e^{-2x}$ is a solution of the differential equation $y^\prime+2y=2e^x$.
\newpage

\item Verify that $y = -t\cos t - t$ is a solution of the initial-value problem 
\[t\diff{y}{t} = y+t^2\sin t \;\;\;\;\;\;\;\;\;\;\;\;\;\; y(\pi)=0\]


\vspace{5pt}
Again, the first step needs to be finding the value of $y^\prime$. But before that, the initial value requirement needs to be addressed. This can be accomplished as follows:
\begin{align*}
y(\pi)&=0\\
-\pi\cos\pi-\pi &= 0\\
&=\pi-\pi\\
&= 0
\end{align*}
Therefore, the initial value is true and the function can be differentiated as follows.
\begin{align*}
y^\prime &= -\diff{}{t}\left(t\cos t+t\right)\\
&= -\left(\diff{}{t}(t)\cos t + \diff{}{t}(\cos t) t\right)-1\\
&= -\cos t +t\sin t-1
\end{align*}
Since the value of $y^\prime$ is now known, simple substitution is all that remains to verify the solution of the problem.
\begin{align*}
y+t^2\sin t &= t\diff{y}{t}\\
&= t\left(-\cos t + t\sin t -1\right)\\
&= -t\cos t+t^2\sin t -t\\
&= \left(-t\cos t-t\right)+t^2\sin t\\
&= y +t^2\sin t
\end{align*}
With this, $y = -t\cos t - t$ has been verified as a solution of the provided initial-value problem.

\newpage 

\item A solution is modeled by the differential equation 
\[\diff{P}{t} = 1.2P\left(1-\dfrac{P}{4200}\right)\]
\begin{enumerate}[label = (\alph*)]
\item For what values is the population increasing?
\item For what values is the population decreasing?
\item What are the equilibrium solutions?
\end{enumerate}


\vspace{5pt}
\begin{enumerate}[label = (\alph*)]
\item This is a very simple problem. The population is increasing when the derivative is positive and vice versa. Therefore, all that needs to be found for this question are the values for which $\diff{P}{t}$ is greater than zero. The eye immediately goes to the expression inside the parentheses, $1-\dfrac{P}{4200}$. It is easily seen that for the derivative to be positive, $P<4200$. The other P in the differential equation creates the lower bound of 0. The values for which the population is increasing are \[(0,4200)\]

\item Using a similar approach to the one used in (a), we just need to find the values of $P$ for which the derivative is negative. First, the P outside of the parentheses ensures that a population below 0 will continually keep decreasing. The expression inside of the parentheses can be used to infer that a population above 4200 will start decreasing because at that point, $1-\dfrac{P}{4200}$ will become negative. The values for which the population is decreasing are \[(-\infty, 0)\cup (4200,\infty)\]

\item We can take the phrase to equilibrium solutions to mean where the value of the derivative is a 0. This only occurs when one of the terms of the differential equation are equal to 0. In this case, the terms are $P$ and $\left(1-\dfrac{P}{4200}\right)$. Some quick mental math results in the values of 0 and 4200 and population values where the population is neither increasing nor decreasing.
\end{enumerate}

\newpage

\item A function $y(t)$ satisfies the differential equation 
\[\diff{y}{t} = y^4-6y^3+5y^2\]
\begin{enumerate}[label = (\alph*)]
\item What are the constant solutions of the equation?
\item For what values of $y$ is $y$ increasing?
\item For what values of $y$ is $y$ decreasing?
\end{enumerate}


\vspace{5pt}
\begin{enumerate}[label = (\alph*)]
\item Using the same approach as the last problem, we simply solve the right hand side for when derivative is in a certain range. For this part, we need to find the solutions for which $\diff{y}{t}$ is 0. This can be solved with some general algebra to yield
\begin{align*}
y^4-6y^3+5y^2 &= y^2(y^2-6y+5)\\
&= y^2(y-5)(y-1)
\end{align*}
Now we can see that the constant solutions are at $y=0, 5, 1$. 

\item To find when $y$ is increasing, the factored form of the differential equation is useful in determining when the derivative is positive. If $y^2(y^2-6y+5)$ is the standard form of the differential equation, $y^2$ can be disregarded because it is always positive. Thus, the only question remaining is when is the $y$ value of the parabola above 0. Because the parabola is positive and the zeroes are at 5 and 1, it is inferred that all value that are not in the interval $(1,5)$ are positive. Thus, $y$ is increasing for the values of $(-\infty,0)\cup(0,1)\cup(5,\infty)$.

\item Using the results from part (b) it is clear that the only values of $y$ for which $y$ is decreasing are $(1,5)$.
\end{enumerate}

\newpage


\item Match the differential equations with the solution graphs labeled I-IV. Give reasons for your choices. 
\end{enumerate}


\end{document}