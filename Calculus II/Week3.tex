\documentclass{article}
\usepackage[english]{babel}
\usepackage[utf8]{inputenc}
\usepackage{fancyhdr}
\usepackage{geometry}
\usepackage{enumitem}
\usepackage{amsmath, amssymb}
\usepackage{graphicx}
\usepackage{float}
\usepackage{gensymb}
\usepackage[thinc]{esdiff}

\geometry{letterpaper, portrait, margin=1in}
\graphicspath{ {images/} }
\pagestyle{fancy}
\fancyhf{}
\lhead{Keerthik Muruganandam}
\rhead{Sep. 29th, 2021 Written Work 3}

\begin{document}

\begin{enumerate}[label=\textbf{(3.\arabic*)}]

\item Give the general solution for the following differential equations:
\begin{enumerate}
\item $4y^{\prime\prime}-12y^\prime+9y=0$
\item $4y^{\prime\prime}-4y^\prime+5y=0$
\item What is the particular solution to part (b) if $y(\pi)=e^{\pi/2}$ and $y^\prime (\pi)=0$.
\end{enumerate}

\begin{enumerate}
\item This is homogeneous, first order DE, so the auxiliary equation is 
\[4r^2-12r+9\]
The solution to this is clearly 
\[r=\frac{3}{2}\]
Because the auxiliary equation has a repeated root, the general solution is 
\[\alpha e^{\frac{3}{2}x}+\beta xe^{\frac{3}{2}x}\]

\item The auxiliary equation for this differential equation is
\[4r^2-4r+5=0\]
The solution to this is	
\[r=\frac{1}{2}\pm i\]
Because the solution is in the form of a complex conjugate, we have to express the solution using the form 
\[y = e^{\frac{1}{2}x}[c_1\cos(x)+c_2\sin(x)]\]

\item Using the first initial condition, we can do the following arithmetic:
\begin{align*}
e^{\pi/2} &= e^{\frac{1}{2}\pi}[c_1\cos(\pi)+c_2\sin(\pi)]\\
&= e^{\pi/2}[-c_1]\\
c_1 &= -1
\end{align*}
We now have the new equation $y = e^{\frac{x}{2}}[c_2\sin(x)-\cos(x)]$. Computing, the derivative with the product rule, we yield
\begin{align*}
\diff{y}{x}&=\diff{}{x}(e^{x/2})[c_2\sin(x)-\cos(x)]+\diff{}{x}[c_2\sin(x)-\cos(x)]e^{x/2}\\
&=\frac{1}{2}e^{x/2}[c_2\sin(x)-\cos(x)]+e^{x/2}[\sin(x)+c_2\cos(x)]
\end{align*}
Substitution solves for $c_2$ like so 
\begin{align*}
0&= \frac{1}{2}e^{\pi/2}[c_2\sin(\pi)-\cos(\pi)]+e^{\pi/2}[\sin(\pi)+c_2\cos(\pi)]\\
&= e^{\pi/2}(\frac{1}{2}-c_2)\\
c_2 &= \frac{1}{2}
\end{align*}
Now, we know $c_2$ so we can determine that the specific solution is
\[y = e^{x/2}[\frac{1}{2}\sin(x)-\cos(x)]\]
\end{enumerate}

\newpage

\item Solve the initial-value problem $x^2\diff{y}{x}-y=2e^{1/x}$, $y(1)=-e$.

First, we must put this first-order linear differential equation into standard form. The standard form is
\[y^{\prime} - \frac{y}{x^2}=\frac{2e^{1/x}}{x^2}\]
To find the integrating factor, we must find the value of $\displaystyle e^{\int -\frac{1}{x^2}\,dx}$. This is integration is simple to complete, and yields $\dfrac{1}{x}$. Thus, the function is transformed into
\[\int\diff{ }{x}(e^{1/x}y)\,dx=\int\frac{2e^{2/x}}{x^2}\,dx\]
To integrate the left hand side, use the $u$-substitution of $\frac{2}{x}$. The value of $du$ is $\dfrac{-2}{x^2}$. The left hand side is transformed into 
\[\int -e^u\,du\]
which is equal to $-e^{2/x}$. The equation is now
\[e^{1/x}y=-e^{2/x}+C\]
Before isolating $y$, it is easier to solve for the value of $C$ like so
\begin{align*}
e^{1/1}(-e)&=-e^{2/1}+C\\
-e^2 &= -e^2+C\\
0 &= C
\end{align*}
Isolating $y$ is simply done.
\begin{align*}
y &= \frac{-e^{2/x}}{e^{1/x}}\\
&= -(e^{\frac{2}{x}-\frac{1}{x}})\\
&= -e^{1/x}
\end{align*}
The particular solution for this initial-value problem is 
\[y=-e^{1/x}\]

\newpage

\item \textbf{Professional Problem: } Chocolate factory tank contains 100 L of pure chocolate. The mixture of crushed Oreos and chocolate is added at 5 L/min. The mixed tank is drained at a rate of 3 L/min. Find and solve the initial value problem. Also, if after 20 minutes, the tank contains 32 kg of Oreo, find the concentration $M$ of Oreos.

We know that the rate of change of the amount Oreos in the tank is $\diff{y}{t}=[\textrm{rate in}]-[\textrm{rate out}]$. The amount of Oreos entering the tank is the concentration of Oreos, $M$, multiplied by the rate at which the mixture enters the tank. The rate out is the concentration of Oreos in the tank multiplied by the rate out. This looks like 
\[\diff{y}{t} = [M\cdot5]-\left[\frac{y}{100+2t}\cdot3\right]\]
The problem in standard form looks like 
\[y^\prime+\frac{3y}{100+2t}=5M\]
To solve, this we need to solve $\displaystyle e^{\int \frac{3}{100+2t}\,dt}$ which is $(100+2t)^{3/2}$. Our new equation is 
\[\diff{ }{t}((100+2t)^{3/2}y) = 5(100+2t)^{3/2}M\]
Solving this by integrating both sides yields
\[(100+2t)^{3/2}y = (100+2t)^{5/2}M+C\]
Using the initial condition $y(0)=0$, we can isolate C. After substitution, we find that 
\[C = -(100)^{5/2}M\]
Isolating $y$ provides the solved equation
\begin{align*}
y &= \frac{\frac{2}{5}(100+2t)^{5/2}M-(100)^{5/2}M}{(100+2t)^{3/2}}\\
&= \frac{\frac{3}{5}(100+2t)^{5/2}M}{(100+2t)^{3/2}}
\end{align*}
To solve the initial value problem of $y(20)=32$, we should isolate $M$. The isolation of $M$ is that 
\[\frac{(100+2t)^{3/2}y}{\frac{3}{5}(100+2t)^{5/2}}=M\]
This is a bit of a hairy equation, so plugging these answers into WolframAlpha is probably the right way to go. Plugging these in provides
\[C = \frac{(100+2(20))^{3/2}(32)}{\frac{3}{5}(100+2(20)^{5/2}}=\frac{8}{21}\]
The concentration of Oreos to melted chocolate is $\frac{8}{21}$ kg/L.
\end{enumerate}


\end{document}

