\documentclass{article}
\usepackage[english]{babel}
\usepackage[utf8]{inputenc}
\usepackage{fancyhdr}
\usepackage{geometry}
\usepackage{enumitem}
\usepackage{amsmath}
\usepackage{graphicx}
\usepackage{tcolorbox}
\usepackage{amssymb}
\usepackage[thinc]{esdiff}
\usepackage{float}

%%%%%%%%%%%%%%%%%%%%%%%%%%%%%%%%%%%%%%%%%%%%%%%%%%%%%%%%%%%%%%%%%%%%%%%%%%%

\geometry{letterpaper, portrait, margin=1in}
\graphicspath{ {images/} }
\pagestyle{fancy}
\fancyhf{}
\lhead{Keerthik Muruganandam}
\rhead{Yadavalli Written Work 6}

%%%%%%%%%%%%%%%%%%%%%%%%%%%%%%%%%%%%%%%%%%%%%%%%%%%%%%%%%%%%%%%%%%%%%%%%%%%

\begin{document}

\begin{enumerate}[label=\textbf{(6.\arabic*)}]

%%%%%%%%%%%%%%%%%%%%%%%%%%%%%%%%%%%%%%%%%%%%%%%%%%%%%%%%%%%%%%%%%%%%%%%%%%%

\item Determine whether $\displaystyle \int_0^1\!\frac{1}{3y-2}\,dy$ is convergent or divergent. If it is convergent evaluate it.
%%%%%%%%%%%%%%%%%%%%%%%%%%%%%%%%%%%%%%%%%%%%%%%%%%%%%%%%%%%%%%%%%%%%%%%%%%%

To find the integral of $\displaystyle \int_0^1\!\frac{1}{3y-2}\,dy$, we can use $u$-substitution with a $u$ of $3y-2$ and a $du=3$. Thus, 
\begin{align*}
    \int_0^1\!\frac{1}{3y-2}\,dy &= \int_0^1\!\frac{1}{3u}\,du
\end{align*}
Computing the antiderivative gives us
\begin{align*}
    \int_0^1\!\frac{1}{3u}\,du 
    &=\left[\frac{1}{3}\ln u\right]_0^1 \\
    &= \left[\frac{\ln(3y-2)}{3}\right]_0^1 \\
    &= \frac{\ln1}{3}-\frac{\ln(-2)}{3} \\
    &= -\frac{\ln(-2)}{3}
\end{align*}
Because our expression for the antiderivative contains $\ln(-2)$ and the natural logarithm is undefined for negative numbers, the integral $\displaystyle \int_0^1\!\frac{1}{3y-2}\,dy$ is divergent.
%%%%%%%%%%%%%%%%%%%%%%%%%%%%%%%%%%%%%%%%%%%%%%%%%%%%%%%%%%%%%%%%%%%%%%%%%%%

\newpage
%%%%%%%%%%%%%%%%%%%%%%%%%%%%%%%%%%%%%%%%%%%%%%%%%%%%%%%%%%%%%%%%%%%%%%%%%%%

\item Use the Comparison Theorem to determine whether $\displaystyle \int_4^\infty\!\frac{5+e^{-x}}{x}\,dx$ is convergent or divergent.
%%%%%%%%%%%%%%%%%%%%%%%%%%%%%%%%%%%%%%%%%%%%%%%%%%%%%%%%%%%%%%%%%%%%%%%%%%%

Let $\displaystyle \int_4^\infty\!\frac{5+e^{-x}}{x}\,dx$ be called $(1)$. Currently $(1)$ is not in a form the Comparison Theorem is able to be used on. To fix that, we can split the integral into two.
\begin{align*}
    \int_4^\infty\!\frac{5+e^{-x}}{x}\,dx &= \int_1^\infty\!\frac{5+e^{-x}}{x}\,dx-\int_1^4\!\frac{5+e^{-x}}{x}\,dx
\end{align*}
Now we can use the Comparison Theorem on the first integral,and compute a finite value for the second integral. We shall compare $\displaystyle \int_1^\infty\!\frac{5+e^{-x}}{x}\,dx$ to $\dfrac{1}{x}$, 
and $1<x<\infty$.
\begin{align*}
    e^{-x} &> 1 \\
    5+e^{-x} &> 1 \\
    \frac{5+e^{-x}}{x} &> \frac{1}{x}
\end{align*}
Now if we use the p-test on $\frac{1}{x}$, we find that $1\not<1$ and therefore it is divergent.\\
Now since
\begin{align*}
    \frac{5+e^{-x}}{x} &> \frac{1}{x}
\end{align*}
the integral $\displaystyle\frac{5+e^{-x}}{x}$ is also divergent. \\
Therefore,
\[\int_1^\infty\!\frac{5+e^{-x}}{x}\,dx-\int_1^4\!\frac{5+e^{-x}}{x}\,dx\]
is divergent which means that
\[\int_4^\infty\!\frac{5+e^{-x}}{x}\,dx\]
is divergent.
%%%%%%%%%%%%%%%%%%%%%%%%%%%%%%%%%%%%%%%%%%%%%%%%%%%%%%%%%%%%%%%%%%%%%%%%%%%

\newpage
%%%%%%%%%%%%%%%%%%%%%%%%%%%%%%%%%%%%%%%%%%%%%%%%%%%%%%%%%%%%%%%%%%%%%%%%%%%

\item Use the Comparison Theorem to determine whether $\displaystyle \int_0^{\pi/2}\!\sec^3x\,dx$ is convergent or divergent.
%%%%%%%%%%%%%%%%%%%%%%%%%%%%%%%%%%%%%%%%%%%%%%%%%%%%%%%%%%%%%%%%%%%%%%%%%%%

Let us compare our integral to the integral of $\sec x$ with the same bounds. Remember that $\displaystyle\int\!\sec x\,dx = \ln|\tan x+\sec x|+C$. Therefore, using our methods for evaluating improper integrals,
\begin{align*}
    \int_0^{\pi/2}\!\sec x\,dx &= \int_0^t\!\sec x\,dx \\
    &= \left[\ln|\tan x+\sec x|\right]_0^t \\
    &= \ln|\tan t+\sec t|-\ln|\tan 0 +\sec 0| \\
    &= \ln|\tan t+\sec t|-\ln1 \\
    &= \ln|\tan t+\sec t|
\end{align*}
Plugging this in to our limit gives us
\begin{align*}
    \lim_{t\to\infty} \ln|\tan t+\sec t| &= \infty \\
    &=\mathrm{Diverges}
\end{align*}
Now we know that $\displaystyle \int_0^{\pi/2}\!\sec x\,dx$ diverges, we can say that for $0<x<\dfrac{\pi}{2}$
\begin{align*}
    \sec x &> 1 \\
    \sec^3x &> \sec x \\
    \displaystyle \int_0^{\pi/2}\!\sec^3x\,dx &> \int_0^{\pi/2}\!\sec x\,dx
\end{align*}
And since we know that $\displaystyle \int_0^{\pi/2}\!\sec x\,dx$ diverges and $\displaystyle \int_0^{\pi/2}\!\sec^3x\,dx>\int_0^{\pi/2}\!\sec x\,dx$, 
we can say using the Comparison Theorem that $\displaystyle \int_0^{\pi/2}\!\sec^3x\,dx$ is divergent.

%%%%%%%%%%%%%%%%%%%%%%%%%%%%%%%%%%%%%%%%%%%%%%%%%%%%%%%%%%%%%%%%%%%%%%%%%%%

\newpage

%%%%%%%%%%%%%%%%%%%%%%%%%%%%%%%%%%%%%%%%%%%%%%%%%%%%%%%%%%%%%%%%%%%%%%%%%%%

\item \textbf{Professional Problem:} Decide whether the following statements are true or false. If a statement is true, prove it and if it's false, provide a specific counterexample.
\begin{enumerate}
    \item If $\int_1^\infty\!f(x)\,dx$ and $\int_1^\infty\!g(x)\,dx$ both converge, then $\int_1^\infty\!f(x)+g(x)\,dx$ also converges.
    \item If $\int_1^\infty\!f(x)+g(x)\,dx$ converges, then $\int_1^\infty\!f(x)\,dx$ and $\int_1^\infty\!g(x)\,dx$ also converge.
    \item If $\int_1^\infty\!f(x)\,dx$ diverges, then $\int_1^\infty\!\left(f(x)\right)^2\,dx$ diverges.
\end{enumerate}
%%%%%%%%%%%%%%%%%%%%%%%%%%%%%%%%%%%%%%%%%%%%%%%%%%%%%%%%%%%%%%%%%%%%%%%%%%%

\begin{enumerate}
    \item Let the statement "If $\int_1^\infty\!f(x)\,dx$ and $\int_1^\infty\!g(x)\,dx$ both converge, then $\int_1^\infty\!f(x)+g(x)\,dx$ also converges" be true. To prove this, first using basic integration
    we can establish that
    \begin{align}
        \int_1^\infty\!f(x)\,dx &= \lim_{x\to\infty} \left(F(x)-F(1)\right)
    \end{align}
    and that $(1)$ converges. Now if we evaluate the integral $\int_1^\infty\!f(x)+g(x)\,dx$, we can complete the math
    \begin{align*}
        \int_1^\infty\!f(x)+g(x)\,dx &= \lim_{t\to\infty} \left[F(x)+G(x)\right]_1^t \\
        &= \lim_{t\to\infty}\left(F(t)+G(t)-F(1)-G(1)\right)
    \end{align*}
    Rearranging the limt produces 
    \begin{align}
        \lim_{t\to\infty}\left(F(t)+G(t)-F(1)-G(1)\right) &= \lim_{t\to\infty} \left(F(t)-F(1)\right)+\lim_{t\to\infty} \left(G(t)-G(1)\right)
    \end{align}
    which using the property in $(1)$ we can clearly see is equal to $\int_1^\infty\!f(x)\,dx + \int_1^\infty\!g(x)\,dx$ and since we know both of those integral converge, we can conclude that 
    $\int_1^\infty\!f(x)+g(x)\,dx$ also converges.
    \item To prove that if $\int_1^\infty\!f(x)+g(x)\,dx$ converges, then $\int_1^\infty\!f(x)\,dx$ and $\int_1^\infty\!g(x)\,dx$ also converge,
    we can take the fact that 
    \begin{align*}
        \int_1^\infty\!f(x)+g(x)\,dx &= \int_1^\infty\!f(x)\,dx + \int_1^\infty\!g(x)\,dx
    \end{align*}
    which was shown in $(2)$ if we convert the limits to integrals and use it to prove that $\int_1^\infty\!f(x)\,dx$ and $\int_1^\infty\!g(x)\,dx$ are convergent since $\int_1^\infty\!f(x)+g(x)\,dx$ converges, each of it's parts must converge
    to produce a finite sum.
    \item A simple counterexample to the statement "If $\int_1^\infty\!f(x)\,dx$ diverges, then $\int_1^\infty\!\left(f(x)\right)^2\,dx$ diverges" is the function $f(x)=\dfrac{1}{x}$. If we apply the $p$-test to $f(x)$ because it is in
    the form $\dfrac{1}{x^p}$, we find that since $f(x)$ has a $p$ of $1$, it fails the p-test and therefore $\int_1^\infty\!f(x)\,dx$ diverges. \\
    However,
    \begin{align*}
        \int_1^\infty\!\left(f(x)\right)^2\,dx &= \int_1^\infty\!\frac{1}{x^2}\,dx
    \end{align*}
    which means that $f(x)^2$ has a $p$ of 2 which is greater than 1 and $\int_1^\infty\!\left(f(x)\right)^2\,dx$ iss convergent. We have just disproved the statement "If $\int_1^\infty\!f(x)\,dx$ diverges, then $\int_1^\infty\!\left(f(x)\right)^2\,dx$ diverges."
    \end{enumerate}
%%%%%%%%%%%%%%%%%%%%%%%%%%%%%%%%%%%%%%%%%%%%%%%%%%%%%%%%%%%%%%%%%%%%%%%%%%%



\end{enumerate}
\end{document}