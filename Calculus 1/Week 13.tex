\documentclass{article}
\usepackage[english]{babel}
\usepackage[utf8]{inputenc}
\usepackage{fancyhdr}
\usepackage{geometry}
\usepackage{enumitem}
\usepackage{amsmath, amssymb}
\usepackage{graphicx}
\usepackage{float}

\geometry{letterpaper, portrait, margin=1in}
\graphicspath{ {images/} }
\pagestyle{fancy}
\fancyhf{}
\lhead{Keerthik Muruganandam}
\rhead{Yadavalli Written Work 13}

\begin{document}

\begin{enumerate}[label=\textbf{(13.\arabic*)}]

%%%%%%%%%%%%%%%%%%% Problem 1		
\item Calculate the radius of convergence and interval of convergence of the series $\displaystyle \sum_{n=1}^\infty \frac{n}{5^n}(x+2)^n$.

To calculate the radius and interval of convergence we will use the Ratio Test. The Ratio Test states that:
\begin{center}
\vspace{-10pt}
\[\lim_{n\to\infty} \left|\frac{a_{n+1}}{a_n}\right|=L\]
If $L>1$, then $a_n$ diverges.
If $L<1$, then $a_n$ absolutely converges.
If $l=1$, then the test was inconclusive.
\end{center}
Applying this test gives us the limit:
\begin{align*}
\lim_{n\to\infty} \left|\frac{a_{n+1}}{a_n}\right| &= \lim_{n\to\infty} \left|\frac{(n+1)(x+2)^{n+1}}{5^{n+1}}\cdot\frac{5^n}{n(x+2)^n}\right|\\
&= \lim_{n\to\infty} \left|\frac{(n+1)(x+2)}{5n}\right|\\
&= \lim_{n\to\infty} \left|\frac{x+2}{5}\right|\cdot\frac{n+1}{n}\\
&= \left|\frac{x+2}{n}\right|
\end{align*}
For the series to converge, 
\[\left|\frac{x+2}{5}\right|<1\]
We can use this to obtain:
\[\left|x+2\right|<5\]
Now we know that the radius of convergence is 5. Solving this inequality gives us two values for $x$: $-7$ and $3$. These are the endpoints for our interval of convergence. Next, we must determine whether the series is convergent there. To do this, we will substitute them into our power series as a replacement for $x$.\\
Substitution of $3$ for $x$ yields the sequence:
\begin{align*}
\frac{n}{5^n}(3+2)^n &=  \frac{n}{5^n}(5)^n\\
&= n
\end{align*}
We can conclude that the interval of convergence ins open-ended at $3$ because the series of $n$ is divergent. Substitution of $-7$ yields:
\begin{align*}
\frac{n}{5^n}(-7+2)^n &=  \frac{n}{5^n}(-5)^n\\
&= (-1)^nn
\end{align*}
We can conclude that this series diverges because it fails the  Test for Divergence, which states that if the limit of a series is not 0, then the series diverges. This series diverges because it oscillates away to $\infty$ on both sides.
\begin{center}
The radius of convergence is 5.\\
The interval of convergence is $(-7,3)$.
\end{center}

%%%%%%%%%%%%%%%%%%% Newpage

\newpage

%Prolem 2	
\item Suppose that $\displaystyle \sum_{n=0}^\infty c_n x^n$ converges when $x=-6$ and diverges when $x=8$. What, if anything can be said about the convergence of the following series? Justify your answer.\\
(a) $\displaystyle \sum_{n=0}^\infty c_n\quad\quad\quad$ (b) $\displaystyle \sum_{n=0}^\infty c_n3^n\quad\quad\quad$ (c) $\displaystyle \sum_{n=0}^\infty c_n(-8)^n\quad\quad\quad$ (d) $\displaystyle \sum_{n=0}^\infty (-1)^nc_n9^n$

\begin{enumerate}
\item First off, we know that the power series is centered at 0, because it only $x$ is being raised to the power of $n$. Using this knowledge we can state that the radius of convergence is at least 6 because when $x=-6$, the series converges and $-6$ is 6 away from 0. In this case, we are using an $x$ of 1, because
\begin{align*}
\sum_{n=0}^\infty c_n &= \sum_{n=0}^\infty c_n1^n
\end{align*}
Since $1<6$, we know that this series converges.
\item Using the same logic as in (a), we can also say that $x=3$ converges because $3<6$ so the proposed series is within the minimum interval of convergence. Therefore, this series converges.
\item For this $x=-8$, we cannot say anything about its convergence. This is because although when $x=8$, the series diverges, our radius of convergence could be 8, which means that this series would be part of an endpoint and could either diverge or converge. Or, it could simply be outside of the interval of convergence. Again, due to the possibility of $x=-8$ being an endpoint on the interval of convergence, we cannot say anything about the convergence or divergence of this series. 
\item We cannot say anything about the convergence or divergence of this series as well. This is because the series is alternating. Although $\displaystyle \sum_{n=0}^\infty c_n9^n$ diverges, due to its $x$ being outside the maximum interval of convergence, since the function is alternating, we don't know if the series as a whole converges or diverges.
\end{enumerate}

%%%%%%%%%%%%%%%%%%% Newpage

\newpage

%%%%%%%%%%%%%%%%%%% Problem 3
\item Define the following power series:
\[f(x)=\sum_{k=0}^\infty x^{2k}=1+x^2+x^4\ldots\quad\quad g(x)=\sum_{k=0}^\infty 3^kx^{2k+1}=x+3x^3+9x^5\ldots\]
\begin{enumerate}
\item Find the interval of convergence and a formula for $f(x)$.
\item Find the interval of convergence and a formula for $g(x)$.
\item Use your answers from (a) and (b) to find a formula and an interval of convergence for $h(x)=1+x+x^2+3x^3+x^4+9x^5\ldots$.
\end{enumerate}

\begin{enumerate}
\item First, let us rearrange the sequence in $f(x)$. We can do this like so:
\begin{align*}
x^{2k} &= \left(x^2\right)^k
\end{align*}
The series is now in the form of a geometric sequence we know how to find the interval of convergence and a formula for it very easily. For a series of the form $\sum a(r)^{n-1}$ to converge, $-1<r<1$. Thus we can create the inequality:
\[-1<x^2<1\]
or,
\[-1<x<1\]
Now that we have our endpoints for the interval of convergence, we need to test the endpoints. For the endpoint of $x=1$ we see that:
\begin{align*}
\sum_{k=0}^\infty 1^{2k} &= \sum_{k=0}^\infty 1
\end{align*}
This obviously goes off to infinity, so the endpoint of $x=1$, diverges. If we take a look at $x=-1$, we see that the sum is
\begin{align*}
\sum_{k=0}^\infty (-1)^{2k} &= \sum_{k=0}^\infty (-1)^{k}
\end{align*}
This endpoint also diverges because it never homes in on a single number, but keeps bouncing between $-1$ and $1$. Therefore, our interval of convergence for $f(x)$ is $(-1,1)$. To find a formula for $f(x)$, we can use the formula for the sum of an infinite geometric series which states that
\begin{align*}
\sum_{k=0}^\infty ar^k=\frac{a}{1-r}
\end{align*}
Using this formula, we can find that
\begin{align*}
\sum_{k=0}^\infty 1\left(x^2\right)^k &= \frac{1}{1-x^2}
\end{align*}
\item To find the interval of convergence and the formula for $g(x)$, we will be using the  same method as in part (a). However, we first have to rearrange the sequence into the form of the geometric series.
\begin{align*}
3^kx^{2k+1} &= x3^k\left(x^2\right)^k\\
&= x\left(3x^2\right)^k
\end{align*}
Now that it is a form which resembles a geometric series, we can plug it into the inequality in part (a).
\[\left|3x^2\right|<1\]
This simplifies to
\[-\frac{1}{\sqrt{3}}<x<\frac{1}{\sqrt{3}}\]
Now we test our endpoints via direct substitution. Substituting $\dfrac{1}{\sqrt{3}}$ provides 
\begin{align*}
\sum_{k=0}^\infty 3^k{\left(\frac{1}{\sqrt{3}}\right)}^{2k+1} &= \sum_{k=0}^\infty \frac{1}{\sqrt{3}}
\end{align*}
This sequence diverges since it's just a constant addition of a positive number. We can use this to predict the divergence of $x=\dfrac{1}{\sqrt{3}}$ because that will just be the same sequence but the constant addition of $-\frac{1}{\sqrt{3}}$. Thus, the interval of convergence is $(-\frac{1}{\sqrt{3}},\frac{1}{\sqrt{3}}$. We can use our rearranged sequence from before and substitute into the sum of a geometric series like so:
\begin{align*}
\sum_{k=0}^\infty x\left(3x^2\right)^k &= \frac{x}{1-3x^2}
\end{align*}
\item From the terms given in the problem, we can say that $h(x)=f(x)+g(x)$. Therefore we can find that 
\begin{align*}
h(x) &= f(x)+g(x)\\
&= \frac{1}{1-x^2}+\frac{x}{1-3x^2}
\end{align*} 
We can reason from this that the interval of convergence for $h(x)$ is values of $x$ for which both $f(x)$ and $g(x)$ converge. The interval where both $f(x)$ and $g(x)$ converge is $\left(-\dfrac{1}{\sqrt{3}},\dfrac{1}{\sqrt{3}}\right)$.
\end{enumerate}

%%%%%%%%%%%%%%%%%%% Newpage

\newpage

%Professional Problem

\item \textbf{Professional Problem:} Consider the power series $\displaystyle \sum_{n=0}^\infty c_n(x-a)^n$ and suppose $c_n\neq0$ for all $n$. Show: if $\displaystyle \lim_{n\to\infty} \left|\frac{c_n}{c_{n+1}}\right|$ exists and is non-negative, then it is equal to the radius of convergence of the power series; and, if $\displaystyle \lim_{n\to\infty} \left|\frac{c_n}{c_{n+1}}\right|=\infty$, the series converges for all $x$. \\
\newline
Let $\displaystyle \lim_{n\to\infty} \dfrac{c_n}{c_n+1}=r$ where $0\le r\le \infty$. If we apply the Ratio Test to the power series $\displaystyle \sum_{n=0}^\infty c_n(x-a)^n$, we receive the result:
\begin{align*}
\lim_{n\to\infty} \left|\frac{c_{n+1}(x-a)^{n+1}}{c_n(x-a)^n}\right| &= \lim_{n\to\infty} \left|\frac{c_{n+1}(x-a)}{c_n}\right|\\
&= \frac{1}{r}\cdot|x-a|
\end{align*}
From this simplification of the Ratio Test we can see that for the series to be convergent,
\begin{align}
\frac{1}{r}|x-a| &< 1
\end{align}
Or in another form:
\begin{align*}
|x-a| &< r
\end{align*}
Thus, we have proved that $\displaystyle \lim_{n\to\infty} \left|\frac{c_n}{c_{n+1}}\right|$ is the radius of convergence. Because $r$ is always a real, non-negative number, we can ensure that there are no imaginary solutions, where $|x-a|$ would have to be less than a negative number. \\
However, there are special cases when $r=\infty$ and $r=0$. When $r=\infty$, we can use the limit law
\begin{center}
If $\displaystyle \lim_{n\to\infty} a_n=+\infty$, then $\displaystyle \lim_{n\to\infty} \frac{1}{a_n}=0$.
\end{center}
In the case of $r=\infty$, we can look back at equation(1) and see that
\begin{align*}
\frac{1}{r}|x-a| &= 0
\end{align*}
We can infer from this that when $r=\infty$, the interval of convergence is all $x$ because $0<1$ always. Conversely, when $r=0$, we can use the limit law
\begin{center}
If $a_n>0$ and $\lim_{n\to\infty} a_n=0$, then $\lim_{n\to\infty} \frac{1}{a_n}=+\infty$.
\end{center}
Using equation (1) once more, we can see that for $r=0$,
\begin{align*}
\frac{1}{r}|x-a| &= \infty
\end{align*}
Infinity is clearly greater than 1 for all $x$, so when $\displaystyle \lim_{n\to\infty} \frac{c_n}{c_n+1}=0$, the power series converges for only one $x$ or does not converge for any $x$.

\end{enumerate}


\end{document}
