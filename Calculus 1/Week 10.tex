\documentclass{article}
\usepackage[english]{babel}
\usepackage[utf8]{inputenc}
\usepackage{fancyhdr}
\usepackage{geometry}
\usepackage{enumitem}
\usepackage{amsmath, amssymb}
\usepackage{graphicx}
\usepackage{amssymb}
\usepackage{float}

\geometry{letterpaper, portrait, margin=1in}
\graphicspath{ {images/} }
\pagestyle{fancy}
\fancyhf{}
\lhead{Keerthik Muruganandam}
\rhead{Yadavalli Written Work 10}

\begin{document}

\begin{enumerate}[label=\textbf{(10.\arabic*)}]

\item If the $n$th partial sum of a series $\displaystyle \sum_{n=1}^\infty a_n$ is $s_n=3-\dfrac{2}{(n+1)^2}$, find $a_n$ and $\displaystyle \sum_{n=1}^\infty a_n$.

To find $a_n$, we will first list out some of the first numbers in the sequence. To do this we can derive that \begin{align}
a_n &= s_n-s_{n+1}
\end{align}
because
\begin{align*}
s_n &= a_n+a_{n-1} \ldots +a_1
\end{align*}
and 
\begin{align*}
s_{n-1} &= a_{n-1}+a_{n-2}\ldots+a_{1},
\end{align*}
therefore
\begin{align*}
a_n &= s_n-s_{n+1}\\
&= (a_n+a_{n-1} \ldots +a_1) - (a_{n-1}+a_{n-2}\ldots+a_{1})\\
&= a_n
\end{align*}
Using $(1)$, we define the first few terms of $a_n$ as
\begin{align*}
a_1 &= \frac{5}{2}\\
a_2 &= \frac{5}{18}\\
a_3 &= \frac{7}{72}\\
a_4 &= \frac{9}{200}
\end{align*}
If we just look at $n>2$, we can define the numerator as $2n+1$. Then, looking at the denominator for the same set of $n$, we find that the denominator is defined as $\dfrac{x^2(x+1)^2}{2}$. We can cover the irregular value of $a_1$ my making it into a piecewise function. Thus we know that
\begin{align*}
a_n &= \begin{cases}
\frac{4n+2}{x^2(x+1)^2}, \text{ if } x > 1\\
\frac{5}{2}, \text{ if } x = 1
\end{cases}
\end{align*}
Now to evaluate $\displaystyle \sum_{n=1}^\infty a_n$, we simply evaluate $\lim_{n\to\infty} 3-\dfrac{2}{(x+1)^2}$.
\begin{align*}
\lim_{n\to\infty} 3-\dfrac{2}{(x+1)^2} &= \lim_{n\to\infty} 3-\lim_{n\to\infty} \frac{2}{(x+1)^2}\\
&= 3-\lim_{n\to\infty} \frac{2}{(x+1)^2}\\
&= 3-\frac{1}{\infty}\\
&= 3
\end{align*}
Thus, $\displaystyle \sum_{n=1}^\infty a_n=3$.

\newpage

\item Find the value of $\displaystyle \sum_{n=1}^\infty \frac{12}{(3n+4)(3n+1)}$ or explain why the series diverges. 

The infinite series $\displaystyle \sum_{n=1}^\infty \frac{12}{(3n+4)(3n+1)}$ is a telescoping series so therefore we will use partial fractions on the sequence inside.
\begin{align*}
\frac{12}{(3n+4)(3n+1)} &= \frac{a}{3n+4} + \frac{b}{3n+1}\\
12 &= a(3n+1)+b(3n+4)\\
&= 3(a+b)n+(a+4b)
\end{align*}
Therefore
\begin{align*}
a+b&=0\\
a+4b&=12\\
\end{align*}
Then we can do the maths
\begin{align*}
a &= -b\\
12 &= 4b-b\\
&= 3b\\
b&= 4\\
a &= -4
\end{align*}
Thus we have the infinite series
\[\sum_{n=1}^\infty\frac{4}{3n+1}-\frac{4}{3n+4}\]
Computing the series gives us
\begin{align*}
\sum_{n=1}^\infty\frac{4}{3n+1}-\frac{4}{3n+4} &= 1-\frac{4}{7}+\frac{4}{7}\ldots-\frac{4}{3n+4}
s_n &= 1-\frac{4}{3n+4}
\end{align*}
Now, we can find the limit of the partial sum to find the value of the infinite series.
\begin{align*}
\lim_{n\to\infty} 1-\frac{4}{3n+4} &= \lim_{n\to\infty}1-\lim_{n\to\infty}\frac{4}{3n+4}\\
&= 1-\lim_{n\to\infty}\frac{4}{3n+4}\\
&= 1-0\\
&= 1
\end{align*}
SO we have found the value of $\displaystyle \sum_{n=1}^\infty \frac{12}{(3n+4)(3n+1)}$ as $1$.

\newpage

\item Show that the sum of the area removed from the Sierpinski carpet is 1.

To start to attempt to find the infinite series that is the area removed from the Sierpinski carpet, we will look at the area removed the first three times.
\begin{figure}[H]
\centering
\includegraphics[scale=1]{sierpinski}
\end{figure}
From this figure, we can see that see that the area removed the first time is $\dfrac{1}{9}$, the second time $\dfrac{8}{81}$, and the third time $\dfrac{64}{729}$.\\
The terms
\[\frac{1}{9}, \frac{8}{81}, \frac{64}{729}\]
matches a geometric sequence of the form
\[\frac{8^{n-1}}{9^n}\]
We can manipulate this into
\begin{align*}
\frac{8^{n-1}}{9^n} &= \frac{8^{n-1}}{9^{n-1+1}}\\
&= \frac{1}{9}\cdot\left(\frac{8}{9}\right)^{n-1}
\end{align*}
This type of infinite series is one we can solve with a partial sum of 
\[\frac{a(1-r^n)}{1-r}\]
Thus, the partial sum is 
\begin{align*}
s_n &= \frac{\frac{1}{9}\left(1-\left(\frac{8}{9}\right)^n\right)}{1-\frac{8}{9}}\\
&= \left(1-\left(\frac{8}{9}\right)^n\right)
\end{align*}
Now, evaluating the limit of our partial sum as it goes to infinity, we get
\begin{align*}
\lim_{n\to\infty} \left(1-\left(\frac{8}{9}\right)^n\right) &= (1-0)\\
&= 1
\end{align*}
Thus we have shown that sum of the area removed from the Sierpinski carpet is equal to 1 which implies the Sierpinski carpet has area 0.


\newpage


\item \textbf{Professional Problem:} 
\begin{enumerate}
\item If $\sum a_n$ is convergent and $\sum b_n$ is divergent, show that the series $\sum(a_n+b_n)$ diverges.
\item If $\sum a_n$ and $\sum b_n$ are both divergent, is $\sum(a_n+b_n)$ necessarily divergent? Prove this, or provide a specific counterexample.
\end{enumerate}

\begin{enumerate}
\item We will prove this statement using proof by contradiction. Assume that there exist sequences $\{a_n\}$ and $\{b_n\}$ such that $\sum a_n$ is convergent and $\sum b_n$ is divergent, and the series $\sum(a_n+b_n)$ is convergent. The series $\sum(a_n+b_n)$ is equal to 
\[a_1+b_1+a_2+b_2\ldots+a_n+b_n\]
Thus, the partial sum of $\sum(a_n+b_n)$ is 
\[s_{a_n}+s_{b_n}\]
Therefore
\begin{align*}
\sum(a_n+b_n) &= \lim_{n\to\infty} s_{a_n}+s_{b_n}\\
&=\lim_{n\to\infty} s_{a_n}+\lim_{n\to\infty} s_{b_n}
\end{align*}
However,
\begin{align*}
\sum b_n &= \lim s_{b_n}\\
&= \text{DNE }
\end{align*}
Then, 
\begin{align*}
\lim_{n\to\infty} s_{a_n}+\lim_{n\to\infty} s_{b_n} &= \lim_{n\to\infty} s_{a_n} + \text{DNE}
\end{align*}
Thus, the series $\sum(a_n+b_n)$ cannot be convergent.

\item A specific counterexample to the statement "\textit{If $\sum a_n$ and $\sum b_n$ are both divergent, $\sum(a_n+b_n)$ is divergent}" is an 
\begin{align*}
\{a_n\}=-1
\end{align*} and a 
\begin{align*}
\{b_n\}=1
\end{align*}
Both of the sums of these sequences diverge since they are adding a constant. However,
\begin{align*}
\{a_n+b_n\} &= 0
\end{align*}
so 
\begin{align*}
\sum(a_n+b_n) &= 0
\end{align*}
since it is just an infinite number of zeroes. Therefore, if $\sum a_n$ and $\sum b_n$ are both divergent, $\sum (a_n+b_n)$ is not necessarily divergent.

\end{enumerate}

\end{enumerate}


\end{document}