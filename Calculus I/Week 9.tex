\documentclass{article}
\usepackage[english]{babel}
\usepackage[utf8]{inputenc}
\usepackage{fancyhdr}
\usepackage{geometry}
\usepackage{enumitem}
\usepackage{amsmath}
\usepackage{graphicx}
\usepackage{amssymb}



\geometry{letterpaper, portrait, margin=1in}
\graphicspath{ {images/} }
\pagestyle{fancy}
\fancyhf{}
\lhead{Keerthik Muruganandam}
\rhead{Yadavalli Written Work 9}

\begin{document}

\begin{enumerate}[label=\textbf{(9.\arabic*)}]

\item Let $a_n=\ln(4n+2)-\ln(n+1)$. Write the first three numbers in the sequence ${a_n}$ and calculate its limit. If the limit doesn't exist, explain why.

Firstly, let's condense the sequence $a_n$. 
\begin{align*}
ln(4n+2)-\ln(n+1) &= \ln\left(\frac{4n+2}{n+1}\right)
\end{align*}
Then we substitute 1, 2, and 3.
\begin{align*}
a_1, a_2, a_3 &= \ln\left(\frac{4+2}{1+1}\right), \ln\left(\frac{8+2}{2+1}\right), \ln\left(\frac{12+2}{3+1}\right)\\
&= \ln3,\ln\frac{10}{3}, \ln\frac{14}{4}
\end{align*}
Next, let us take the limit of $\ln\left(\frac{4n+2}{n+1}\right)$ as $n$ approaches $\infty$. First we must change $a_n$ into $f(x)$ and take the limit 
\begin{align*}
\lim_{n\to\infty} a_n &= \lim_{x\to\infty} \ln\left(\frac{4x+2}{x+1}\right)
\end{align*}
We can use the limit law that states that "If $\lim_{n\to\infty}a_n=L$ and $f(x)$ is continuous at $L$, $\lim_{n\to\infty}f(a_n)=f(L)$" to rearrange our limit and make it easier to evaluate.
\begin{align*}
\lim_{x\to\infty} \ln\left(\frac{4x+2}{x+1}\right) &= \ln\left(\lim_{x\to\infty} \frac{4x+2}{x+1}\right)
\end{align*}
This is relatively simple to evaluate as we simply divide everything within the limit by $x$.
\begin{align*}
\ln\left(\lim_{x\to\infty} \frac{4x+2}{x+1}\right) &= \ln\left(\lim_{x\to\infty} \frac{4+\frac{2}{x}}{1+\frac{1}{x}}\right)\\
&= \ln\left(\frac{4+0}{1+0}\right)\\
&= \ln4
\end{align*}
Thus, the limit of the sequence $a_n$ is $\ln4$. 

\newpage


\item A sequence $b_n$ is given by $b_1=2$, $b_{n+1}=\sqrt{6+b_n}$.
\begin{enumerate}
\item Use induction to prove ${b_n}$ is increasing.
\item Use induction to prove ${b_n}$ is bounded.
\item Explain why Monotonic Sequence Theorem applies to ${b_n}$.
\item Find $\lim_{n\to\infty}b_n$.
\end{enumerate}


\begin{enumerate}
\item First let us do the Base Case.
\begin{align*}
b_2 &= \sqrt{6+b_1}\\
&= \sqrt{6+2}\\
&= \sqrt{8}\\
&= 2\sqrt{2}>b_1
\end{align*}
Thus we have seen that for $n=1$, the sequence is indeed increasing. For our assumption, let us assume that the sequence is increasing for $n=k$ and that 
\begin{align}
b_{k} &\ge b_{k-1}
\end{align}
Then, we shall show that this is true for $n=k+1$ and that:
\begin{align*}
b_{k+1} &\ge b_k
\end{align*}
Now substituting our $b_k$ in we can do the math:
\begin{align*}
b_{k+1} &\ge b_k\\
\sqrt{6+b_k} &\ge \sqrt{6+b_{k-1}}\\
6+b_k &\ge 6+b_{k-1}\\
b_{k} &\ge b_{k-1}
\end{align*}
Looking back at $(1)$, note that we assumed this inequality to be true, and therefore, $b_n$ is increasing.
\item Let the bound of $b_n$ be 3. Then, our base case is
\begin{align*}
b_1 &= 2\\
&\le 3
\end{align*}
Thus, we shall assume that $b_k\le3$ for $n=k$, and prove that $b_{k+1}\le3$. We know that 
\begin{align*}
b_{k+1} &= \sqrt{6+b_k}
\end{align*}
Using our assumption that $b_k\le3$, we can say that 
\begin{align*}
\sqrt{6+b_k} &\le \sqrt{6+3}
&=\sqrt{6+b_k} &\le 3
3 &\le3
\end{align*}
Thus we have proved that $b_n$ is bounded.
\item The Monotonic Sequence Theorem applies to $b_n$ since we have proved that $b_n$ is both monotonic and bounded, which the two requirements for the Monotonic Sequence theorem and thus $b_n$ converges.
\item First of all note that for our sequence
\begin{align}
\lim_{n\to\infty} b_n &= \lim_{n\to\infty} b_{n+1}
\end{align}
Let
\begin{align*}
B &= \lim_{n\to\infty} b_n
\end{align*}
Then, we can do the algebra using (1) that 
\begin{align*}
B &= \lim_{n\to\infty} b_{n+1}\\
&= \lim_{n\to\infty} \sqrt{6+b_n}
\end{align*}
Then, using limit laws, we can change our limit above to:
\begin{align*}
\lim_{n\to\infty} \sqrt{6+b_n} &= \sqrt{6+\lim_{n\to\infty} b_n}\\
&= \sqrt{6+B}\\
B= \sqrt{6+B}\\
B^2 &= 6+B\\
0 &= B^2-B-6\\
&= \left(B-3\right)\left(B+2\right)
\end{align*}
From this we can see that $B=3,-2$, however since $b_n$ is increasing and $-2<b_1$, $B=3$.\\
Therefore $\lim_{n\to\infty} b_n=3$
\end{enumerate}

\newpage

\item Let $c_n=\dfrac{3-2n}{n}$.
\begin{enumerate}
\item Prove that ${c_n}$ is decreasing w/ algebra
\item Use algebra to prove that ${c_n}$ is bounded below by $-2$ and above by 1.
\end{enumerate}

\begin{enumerate}
\item If $c_n$ is decreasing, then
\begin{align*}
\frac{3-2(n+1)}{n+1} &\le \frac{3-2n}{n}\\
0 &\le \frac{3-2n}{n} - \frac{3-2(n+1)}{n+1}\\
&\le (n+1)(3-2n)-n(1-2n)\\
&\le 3n-2n^2+3-2n-n+2n^2\\
&\le 3	
\end{align*}
Thus, we have proved that since the value of $n$ has no bearing on the relationship between $c_n$ and $c_{n+1}$ and our proposed inequality was true, ${c_n}$ is decreasing.
\item Since $c_1=1$ and the sequence is decreasing, we know that it is bounded above by 1. To prove that the sequence is bounded below by $-2$, we can find the range of $y$ like so:
\begin{align*}
\frac{3-2x}{x} &= y\\
3-2x &= xy\\
3 &= xy+2x\\
&= x(2+y)\\
\frac{3}{2+y} &= x
\end{align*} \\
From this, we can clearly see that the range of $c_n$ is ${1,-2}$ since the domain is ${1,\infty}$.
\end{enumerate}

\newpage


\item \textbf{Professional Problem:} Prove or provide a specific counterexample to the following statements;
\begin{enumerate}
\item If ${|a_n|}$ is convergent, ${a_n}$ is convergent.
\item If ${a_n}$ and ${b_n}$ are divergent, then ${b_n+a_n}$ diverges.
\item If ${a_n}$ and ${b_n}$ are divergent, then ${a_n\cdot b_n}$ diverges.
\item If ${a_n}$ and ${a_n\cdot b_n}$ are convergent sequences, then ${b_n}$ converges.
\end{enumerate}

\begin{enumerate}
\item \textbf{This statement is false.} The sequence $a_n=(-1)^n$ is a counterexample to the statement
\begin{center}
"If ${|a_n|}$ is convergent, ${a_n}$ is convergent."
\end{center}
This is because $(-1)^n$ is divergent by the theorem in the textbook which states,
\begin{center}
"The sequence ${r^n}$ is convergent if $-1<r\le1$ and divergent for all other values of $r$."
\end{center}
However, ${|a_n|}=1$ because of the limit law 
\begin{align}
\lim_{x\to\infty} c &= c
\end{align}
Therefore, the statement is false.

\item \textbf{This statement is false.} The sequences \[a_n=n\] and \[b_n=-n\] are counterexamples to the statement 
\begin{center}
"If ${a_n}$ and ${b_n}$ are divergent, then ${b_n+a_n}$ diverges."
\end{center}
This is because they are both divergent on their own, which can be found by simply evaluating the limits of both, but $a_n+b_n=0$. \\
According to the limit law referenced in equation $(1)$, $a_n+b_n$ is therefore divergent since the sequence is a constant.

\item \textbf{This statement is false.} The sequences of \[a_n=\cos n\] and \[b_n=\sec n\] are counterexamples to the statement
\begin{center}
"If ${a_n}$ and ${b_n}$ are divergent, then ${a_n\cdot b_n}$ diverges."
\end{center} 
They diverge because evaluation of them using the theorem
\begin{center}
"If $\lim_{n\to\infty} a_n = L$ and the function $f$ is continuous at $L$, then $\lim_{n\to\infty}f(a_n)=f(L)$.
\end{center}
requires the evaluation of $\cos\infty$ and $\sec\infty$. Therefore, $a_n$ and $b_n$ are divergent since they require the evaluation of $\infty$, $a_n\cdot b_n=1$ and again using equation (1), is convergent.

\item \textbf{This statement is false.} The combination of the sequences $a_n=0$ and $b_n=n$ is a counterexample of the statement
\begin{center}
"If ${a_n}$ and ${a_n\cdot b_n}$ are convergent sequences, then ${b_n}$ converges."
\end{center}
According to (1), then the limit is convergent. However, $b_n=n$ isn't convergent on its own which can be found from basic evaluation of the limit.  
\end{enumerate}

\end{enumerate}


\end{document}